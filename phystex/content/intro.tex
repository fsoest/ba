% !TeX root = ../BA_main_englisch.tex
% !TeX spellcheck = en_GB
Due to the increasing availability of data and computing power, machine learning methods have become a standard approach in many fields such as natural language processing \cite{DBLP:journals/corr/VaswaniSPUJGKP17}.
Additionally, these methods are being applied to a broad range of problems in the physical sciences, from statistical physics to quantum computing \cite{Carleo_2019, wise2021using}.

Machine learning has also found use in the field of energy harvesting, concerned with extracting energy from external excitations, e.g. vibrations in human motion \cite{Liu2019}.
In this work we extend the concept to the quantum case.
We examine a system that can be driven by an arbitrary excitation, modelled as a time-dependent system Hamiltonian.
The Transducer is modelled in a similar fashion and can extract work from the system following the framework in \cite{beyer2020}.


The remainder of this work is structured as follows.
In Chapter \ref{background} we review the collision model framework used in our approach and introduce two machine learning architectures.
We investigate the system under consideration in Section \ref{dep_dt} and apply the aforementioned architectures in Sections \ref{n_2_ml} to \ref{work_cost}.
We summarise our findings and provide an outlook in Chapter \ref{outlook}.