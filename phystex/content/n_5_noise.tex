% !TeX spellcheck = en_GB
\subsubsection{Noise resistance}
Besides the efficiency of the networks the resistance of their predictions to noise is also of interest.
We create a noisy sequence $\{\ket{\phi_j}\}$ of N qubits from $\{\ket{\psi_j}\}$ using
\begin{align*}
	\ket{\phi_j} = e^{-i H_j \tau} \ket{\psi_j}, \ \forall j \in [1, N].
\end{align*}
$H_j$ are randomly generated Hermitian matrices and $\tau$ is a real parameter used to control the strength of the noise.
To quantify the dissimilarity between $\{\ket{\phi_j}\}$ and $\{\ket{\psi_j}\}$ we use the fidelity $F$ as defined in \cite{10.5555/1972505}:
\begin{align*}
	F_{\text{run}}(\{\ket{\phi_j}\}, \{\ket{\psi_j}\}) = \prod_j F(\ket{\phi_j}, \ket{\psi_j}) = \prod_j \abs{\bra{\phi_j}\ket{\psi_j}}.
\end{align*}

The performance of the bidirectional LSTM on noisy sequences is presented in Figure \ref{noisedt5} 

\begin{figure}
	\centering
	\begin{subfigure}{0.4\textwidth}
		\centering
		\includegraphics[width=\textwidth]{img/noisy_drive_bi_true_3}
		\subcaption{}
	\end{subfigure}
	\begin{subfigure}{0.4\textwidth}
		\centering
		\includegraphics[width=\textwidth]{img/noisy_trans_bi_true_3}
		\subcaption{}
	\end{subfigure}
	\caption{We plot the difference $\Delta W = \overline{W}_{noise} - W_{pred}$ of each element of the test set for the bidirectional LSTM and $\Delta \mathrm{T} = 5$. $W_{pred}$ is the work output following the model prediction. \textbf{(a)} We create 100 noisy Drive sequences and calculate the average $\overline{W}_{noise}$ of their work output following the predicted Transducer protocol. \textbf{(b)} We create 100 noisy Transducer sequences and calculate the average of their work output with a given Drive sequence from the test set. In both plots, the black dots indicate the averaged fidelities and $\Delta W$ for a given $\tau$.}
	\label{noisedt5}
\end{figure}