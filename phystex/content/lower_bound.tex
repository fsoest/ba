% !TeX root = ../BA_main_englisch.tex
% !TeX spellcheck = en_GB
\section{Local optimisation protocol}
In this Chapter we derive a lower bound on the extractable work given a drive sequence $\{\ket{\psi_D^i}\}$ when the system state $\rho_S$ is initialised in the eigenstate $\ket{+}\bra{+}$ with higher eigenvalue of the partial system Hamiltonian $H_{DS} = \bra{\psi_D}\bra{\psi_T} H_I \otimes \mathds{1}_T \ket{\psi_D} \ket{\psi_T}$.
By following a policy where the work output of each step $dW$ is optimised locally, we find that the transducer can be chosen such that $dW \geq 0$ for all drives.

We start in an eigenstate of the drive Hamiltonian, without loss of generality let $H_{DS} \propto \sigma_x$.
By local optimisation we find $H_{ST}' = -\frac{\sigma_x}{2}, H_{ST} = \frac{\sigma_x}{2}$.
This extracts the maximal $dW = 1$.

We now examine the following step and generalise to an arbitrary state with zero y-component $\rho_S = (\cos(\theta/2) \ket{0} + \sin(\theta/2) \ket{1})(\cos(\theta/2) \bra{0} + \sin(\theta/2) \bra{1})$.
Note that its projection onto the x-y-plane on the Bloch sphere is anti-parallel to the transducer, which is determined by the previous optimisation as $H_{ST} = -\frac{\sigma_x}{2}$.
The total system Hamiltonian is given by
\begin{align*}
H &= H_{DS} + H_{ST} =  \delta \sigma_+ + \delta^* \sigma_{-} - \frac{\sigma_x}{2} =: \alpha \sigma_+ + h.c. = \Re{\alpha} \sigma_x + \Im{\alpha} \sigma_y, \\
\delta &= \sin{\theta_D} e^{i\phi_D} / 2.
\end{align*}
The evolved state $\rho'$ can be stated as (see Appendix \ref{deriv_jump})
\begin{align*}
\rho_S' &= e^{-iH\Delta \mathrm{T}} \rho_S \ e^{iH\Delta \mathrm{T}} \\
&=\frac{1}{2} (\mathds{1} + \vec{r} \cdot \vec{\sigma}),       
\end{align*}
with
\begin{align*}
\vec{r} &= ( \ \Re{a}, \ \Im{a}, \ b/2 + \cos{(2 \alpha \Delta \mathrm{T})} \cos{\theta} \ ), \\
a &= -\frac{\alpha}{\abs{\alpha}} i \ \sin(2 \abs{\alpha} \Delta \mathrm{T}) \cos{\theta} + \frac{\alpha}{\alpha^*} \sin{\theta} \ \sin^2(\abs{\alpha} \Delta \mathrm{T}) + \sin{\theta} \ \cos^2(\abs{\alpha} \Delta \mathrm{T}), \\
b &= -\frac{1}{\alpha} \sin{(2 \abs{\alpha} \Delta \mathrm{T})} \sin{(\theta)} \Im{\alpha}.
\end{align*}
The step work output $dW$ is then given by
\begin{align*}
dW &= \Tr{\rho' (H_{ST} - H'_{ST})} = \Tr{\rho' \left(-\frac{\sigma_x}{2} - \Re{\tau} \sigma_x - \Im{\tau} \sigma_y \right)} \\
&= -\frac{1}{2} \Re{a} - \Re{\tau}\Re{a} - \Im{\tau} \Im{a}, \\
\tau &= \frac{1}{2} \sin{\theta_T} e^{i\phi_T}. 
\end{align*}
$dW$ has a maximum for $\theta_T = \frac{\pi}{2}, \phi_T = \arctan{\frac{\Im{a}}{\Re{a}}} + \pi$, giving
\begin{align*}
dW_{lo} = \frac{1}{2}\left(\sqrt{\Re{a}^2 + \Im{a}^2} -\Re{a}\right) \geq 0 \ \forall a.
\end{align*}
$H'_{ST}$ is again anti-parallel to the projection of $\rho'_S$ onto the x-y-plane and thus the prerequisite (after a passive transformation) for the above is given.

$a$ only depends on the current $\ket{\psi_D}$ and $\rho_S$.
For a given drive sequence the sum of the locally optimised work outputs $dW_{lo}$ provides a lower bound on the total extractable work as each contribution is non-negative. In the following chapter we will show that protocols with higher outputs exist by optimising globally.