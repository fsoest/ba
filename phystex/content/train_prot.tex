% !TeX spellcheck = en_GB
In all models we split the data into training, validation and test data with $p_{test} = 18 \%$ and $p_{valid} = 8.2 \%$. 
The models are implemented using the PyTorch library \cite{NEURIPS2019_9015}. 
Training is performed over $n$ epochs, in each of which the complete training set is used once in batches of \textit{batch size}.
In each batch, the gradient of the loss function with regard to the model parameters is calculated.
The gradient average over the data points in a batch are used to update the trainable parameters.
We use early stopping to stop training when the value of the cost function on the validation set does not improve for \textit{patience} epochs.
We apply dropout \cite{hinton2012improving} after all layers except for input and output.

\section{Hyperparameters Section \ref{n_2_ml}}
For $N=2$, we use the Adam optimiser with $\beta_1 = 0.9$, $\beta_2 = 0.98$ and $\epsilon = 10^{-9}$. The learning rate (LR) is determined by an exponential decay schedule
\begin{equation} \label{eds}
\mathrm{LR} \ (i) = \mathrm{initial \ LR} * \mathrm{decay \ rate}^i,
\end{equation}
where $i$ denotes the optimiser step.


\begin{table}[h]
	\centering
	\begin{tabular}{c | c | c }
		Patience & Initial LR & Decay Rate \\
		\hline
		100 & $10^{-2}$ & 0.995 \\
	\end{tabular}
	\caption{Hyperparameters used in training for the models in section \ref{n_2_ml}.}
	\label{hyperparams_n_2}
\end{table}

\section{Hyperparameters Sections \ref{n_5_ml} and \ref{n_5_dt1}}
The learning rate (LR) is determined by the `Reduce LR on Plateau' schedule.
The learning rate is multiplied by the hyperparameter \textit{factor} when the cost function does not improve for \textit{patience} / \textit{pat drop} epochs.
The hyperparameters are found using Bayesian optimisation implemented by \cite{wandb} and can be found in Table \ref{hyperparams_n_5}.
The FCANN in Section \ref{n_5_ml} has three hidden layers with 2000 neurons each.
The hyperparameters for the bidirectional LSTM from Section \ref{work_cost} are presented in Table \ref{hyperparams_work_cost}.

\begin{table}[h]
	\centering
	\begin{tabular}{l | l}
		Hyperparameter & Value \\
		\hline
		Optimiser & SGD \\
		Batch size & 44 \\
		Dropout & 0.3179732914255167 \\
		Input layer 1 size & 793 \\
		Input layer 2 size & 488 \\
		Output layer size & 228 \\
		LSTM hidden size & 324 \\
		Initial learning rate & 0.02847560288866327 \\
		LSTM layers & 3 \\
		Pat drop & 3.698498258242224 \\
		Patience & 55 \\
		Factor & 0.24509238889070978
	\end{tabular}
	\caption{Hyperparameters used for the models in Sections \ref{n5} and \ref{work_cost}.}
	\label{hyperparams_n_5}
\end{table}

\begin{table}[h]
	\centering
	\begin{tabular}{l | l}
		Hyperparameter & Value \\
		\hline
		Optimiser & SGD \\
		Batch size & 23 \\
		Dropout & 0.1343610845377841 \\
		Input layer 1 size & 58 \\
		Input layer 2 size & 1115 \\
		Output layer size & 738 \\
		LSTM hidden size & 528 \\
		Initial learning rate & 0.0484126686015994 \\
		LSTM layers & 1 \\
		Pat drop & 1.5006100847416177 \\
		Patience & 78 \\
		Factor & 0.5967308787201352
	\end{tabular}
	\caption{Hyperparameters used for the models in Sections \ref{n5} and \ref{work_cost}.}
	\label{hyperparams_work_cost}
\end{table}