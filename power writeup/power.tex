\documentclass{article}
\usepackage{ bbold }
\usepackage{physics}
\usepackage{ wasysym }
\usepackage{amsmath}
\usepackage{dsfont}
\begin{document}
% !TeX root = ../BA_main_englisch.tex
% !TeX spellcheck = en_GB
Before we apply neural networks, we derive a lower bound on the extractable work given a drive sequence $\{\ket{\psi_D^i}\}$.
By following a policy where the work output of each step $dW$ is optimised locally, we find that the transducer can be chosen such that $dW \geq 0$ for all drives.

We start in an arbitrary system state $\rho_S^n$.
Consider the evolved state $\rho_S^{n+1} =\frac{1}{2} (\mathds{1} + \vec{r} \cdot \vec{\sigma})$ after time $\Delta \mathrm{T}$, with $\vec{r} = ( x, y, z )$, where $\vec{r}$ would be determined by a unitary transformation in our case (note that the following is valid for any CPTP operation). The step work output $dW$ is then given by
\begin{align*}
dW &= \Tr{\rho' (H_{ST}^{n} - H_{ST}^{n+1})} \\
&= \Tr{\rho' \left((\Re{\tau^n} - \Re{\tau^{n+1}}) \sigma_x + (\Im{\tau^n} - \Im{\tau^{n+1}} ) \sigma_y \right)} \\
%&= -\frac{x}{2} - \Re{\tau} x - \Im{\tau} y, \\
\tau^n &= \frac{1}{2} \sin{\theta_T^n} e^{i\phi_T^n}. 
\end{align*}
$dW$ has a maximum for $\theta_T = \frac{\pi}{2}, \phi_T = \arctan{\frac{y}{x}} + \pi$, giving
\begin{align*}
dW_{lo} = \frac{1}{2}\left(\sqrt{x^2 + y^2} -x\right) \geq 0 \ \forall x, y.
\end{align*}
$H'_{ST}$ is again anti-parallel to the projection of $\rho'_S$ onto the x-y-plane and thus the prerequisite (after a rotation about the z-axis) for the above is given.

$\vec{r}$ only depends on the current $\ket{\psi_D}$ and $\rho_S$.
For a given drive sequence the sum of the locally optimised work outputs $dW_{lo}$ provides a lower bound on the total extractable work as each contribution is non-negative. In the following chapter we will show that protocols with higher outputs exist by optimising globally over all time steps.
\end{document}